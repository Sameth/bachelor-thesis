\chapter*{Záver}
\addcontentsline{toc}{chapter}{Záver}

V tejto práci sme sa zaoberali indexáciou potenciálne veľkej sady
čítaní. Základným prístupom k problému bolo nájdenie $k$-nadslova
k čítaniam a vytvorenie pomocných polí a štruktúr, aby sme vedeli
odpovedať na dotazy o pôvodných čítaniach.

Popísali sme si spôsob, ako nájsť zaručene najkratšie $k$-nadslovo
v polynomiálnom čase, pokiaľ je nutný podgraf $k$-grafu čítaní
súvislý. Ďalej sme si popísali spôsob, ako vieme hľadať $k$-nadslovo
aj v nesúvislom $k$-grafe, pričom toto $k$-nadslovo bude dlhšie
od najkratšieho možného o najviac $(k-1)$-násobok počtu komponentov.
Taktiež sme si popísali, ako vieme s využitím náhody nájsť nejaké
o málo dlhšie nadslovo v čase lineárne závislom od veľkosti $k$-grafu.

Pri indexácii sme si popísali tri základné typy dotazov, na ktoré vieme
s využitím $k$-nadslova odpovedať, zároveň spomenuté dotazy považujeme
za rozumné pokrytie dotazov, ktoré sa s pomocou $k$-nadslova dajú spracovávať.
Podobne ako spomenuté dotazy vieme riešiť aj zvyšné dotazy spomenuté v
článku N. Philippe a kol. \cite{gk_arrays}, spomenuté v prvej kapitole
tejto práce:
\begin{enumerate}
    \item Túto otázku sme riešili priamo v práci.
    \item Túto otázku vieme riešiť podobne, ako počet výskytov podreťazca
          v čítaniach, stačí odlišne vypočítať pole počtov výskytov.
    \item Táto otázka sa dá zodpovedať pomocou rovnakých štruktúr, ako prvá.
          Pri počítaní si nebudeme ukladať iba, ktoré čítanie sme práve našli,
          uložíme si aj pozíciu v ňom.
    \item Túto otázku sme riešili piamo v práci.
\end{enumerate}

Zvyšné otázky sú len modifikáciou prvých štyroch, keď nás zaujímajú iba čítania,
ktoré daný podreťazec obsahujú iba raz. Tie vieme spracovať úplne rovnako, stačí
vhodne prispôsobiť polia a trošku inak zapracovať čítania do pomocných polí.

Nakoniec samotné testovanie našej štruktúry ukázalo, že naša štruktúra je pre
väčšie $k$ na nejakých dátach efektívnejšia než CR-index \cite{cr_index}, zatiaľčo pre menšie $k$
je od neho horšia. Na niektorých dátach sa ukazalo, že naša štruktúra síce zaberá
o niečo viac pamäte, ale dotazy vie spracovávať rýchlejšie.

V tejto práci sme neriešili chyby v čítaniach, keďže vieme odhadnúť, že každá
chyba v dátach môže $k$-nadslovo predĺžiť najviac o $2*k - 1$. Keďže je však
naša štruktúra efektívnejšia pre väčšie $k$, riešenie chýb môže patriť
k nevyhnutným krokom k použiteľnosti našej štruktúry pre relatívne veľké hodnoty $k$.

Na druhú stranu, s riešením chýb podobným ako v CR-indexe prichádzajú aj skryté úskalia. Jedným z nich
je nutnosť pri každom dotaze na podreťazec $r$ spracovať viacero reťazcov,
ktoré mohli vzniknúť ako oprava nejakého čítania, ktoré pôvodne $r$ obsahovalo.
Druhým z nich je odpovedanie na otázky o počte výskytov. Keďže niektoré $k$-tice
sú uložené v pozmenenej podobe, je nutné aj pri počítaní overovať jednotlivé
čítania.

Ďalším problémom do budúcnosti je otázka, či neexistuje ešte iný spôsob, ako
sformulovanť nadslovo, s pomocou ktorého by sa dali implementovať
ešte efektívnejšie štruktúry. Jedna možná formulácia je nadslovo so skóre,
v ktorom si môžeme dovoliť rozdeliť čítanie na dve časti za cenu nejakého
záporného skóre, do ktorého sa premietne nutnosť nejakým spôsobom uložiť
informáciu o tomto rozdelení.
