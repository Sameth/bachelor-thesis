\chapter*{Úvod}
\addcontentsline{toc}{chapter}{Úvod}

S vývojom rozmanitých biologických technológií sa začalo
pri skúmaní organizmov objavovať nevídané množstvo dát. Kvôli ich
enormnému množstvu, špecifickým vlastnostiam a nárokom na spracovanie
vznikol odbor bioinformatika.

Podstatná skupina dát sa zaoberá DNA organizmov. V tejto práci sa
budeme venovať čítaniam, krátkym podreťazcom DNA, ktoré vieme presne
zisťovať počas hľadania celej DNA nejakého organizmu. Aby sme vedeli
rýchlo zisťovať mnohé zaujímavé skutočnosti z týchto reťazcov, je
nutné mať štruktúru, ktorá dokáže rýchlo odpovedať na rôzne otázky
o nich. Pre mnohé účely stačí, aby sme vedeli rýchlo odpovedať na
otázky o nejakých kratších podreťazcoch týchto čítaní.

Našim prístupom k tomuto problému bude vytvorenie vhodného nadslova
čítaní, ktoré bude obsahovať všetky ich podreťazce dĺžky $k$ (pre vopred dané $k$).
Pomocou niekoľkých podporných štruktúr potom
budeme vedieť odpovedať na otázky o podreťazcoch dlhých presne $k$.

V prvej kapitole sa oboznámime so základnými biologickými znalosťami a pojmami,
z ktorých vychádza táto práca. Podrobnejšie si popíšeme motiváciu a ciele
práce, spomenieme si predošlé riešenia k tomuto problému a zadefinujeme
zopár pojmov, ktoré budeme neskôr používať.
V druhej kapitole sa budeme venovať hľadaniu spomenutého nadslova,
spomenieme si zopár teoretických výsledkov o tomto probléme
a popíšeme, ako sa dá toto hľadanie implementovať.
V tretej kapitole sa pozrieme, ako a s pomocou akých štruktúr budeme
vedieť vďaka nájdenému nadslovu odpovedať na rôzne otázky.
V štvrtej kapitole sa pozrieme, aké sú nároky na pamäť a čas našej
implementácie a porovnáme si ju s inou prácou v oblasti.

Súčasťou práce je aj ukážková implementácia v jazyku C++.
