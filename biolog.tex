\chapter{Základné pojmy, motivácia a súvisiace práce}

V tejto kapitole čiatateľa oboznámime so základnými biologickými
a informatickými pojmami, s ktorými sa môže stretnúť v tejto práci, s motiváciou
pre riešenie tohto problému a s inými, súvisiacimi prácami v tejto oblasti.

\section{Biologická motivácia}

Genetická informácia všetkých známych bunkových organizmov je uložená v
deoxyribonukleovej kyseline, skrátene DNA. Ide o zložitú molekulu, pozostávajúcu
z mnohých menších častí, ktoré sa nazývajú nukleotidy. Každý nukleotid pozostáva
z fosfátového zvyšku, deoxyribózy a dusíkatej bázy. Nukleotidy sa od seba líšia
v dusíkatej báze, ktorá môže byť štyroch typov: adenín, guanín, tymín, cytozín.

Celá genetická informácia je kódovaná v podstate iba usporiadaním nukleotidov,
pre krátkosť sa toto usporiadanie zapisuje reťazcom pozostávajúcim
zo znakov \verb_A_, \verb_G_, \verb_T_ a \verb_C_, každý znak zodpovedajúci
príslušnej dusíkatej báze. Celá genetická informácia jedného organizmu sa nazýva
genóm. Proces zisťovania genómu, čiže postupnosti dusíkatých báz, resp.
znakov, sa nazýva sekvenovanie.

\subsection{Sekvenovanie}

Väčšina sekvenovacích postupov má spoločné črty. Pomocou zložitých chemických
a fyzikálnych procesov sa najprv zistí mnoho krátkych reťazcov, ktoré sa
v reťazci DNA nachádzajú na vopred neznámych miestach. Tieto krátke reťazce
sa nazývajú \emph{čítania}, z anglického \emph{read}.

Z týchto čítaní sa potom pomocou rôznych algoritmov zisťuje, aký bol celý reťazec DNA.
Tento proces, nazývaný skladanie sekvencií, je pomerne náročný, keďže čítania sa nachádzajú na neznámych miestach
v DNA a navyše, väčšina metód na ich zistenie má malú šancu pomýliť sa pri určovaní
jedného nukleotidu. Preto sa pri každej metóde sekvenovania zisťuje toľko čítaní, aby
v priaznivom prípade mnohonásobne pokryli celý reťazec DNA. Očakávaný počet, koľkokrát
sme pokryli jeden konkrétny nukleotid týmito čítaniami, sa nazýva \emph{pokrytie},
z anglického \emph{coverage}.

\subsection{Indexácia čítaní}

Pokiaľ máme k dispozícii genóm nejakého jedinca, môže nás zaujímať, koľko, prípadne
ktoré čítania sa prekrývajú s nejakou časťou. Pomocou takýchto dotazov vieme zisťovať
mnohé zaujímavé informácie. Jednou možnosťou je napríklad hľadať rozdiely medzi
jedincami jedného druhu, inou je zisťovať, ktoré časti DNA majú akú funkciu, prípadne
ktoré majú vôbec nejakú. Pre prípady, kedy máme k dispozícii referenčný genóm živočíšneho
druhu, existujú efektívne štruktúry, ktoré vedia na dané dotazy odpovedať.

V mnohých prípadoch však referenčný genóm k dispozícii nemáme, napriek tomu by sme
radi vedeli odpovedať na podobné otázky. Jedno konkrétne uplatnenie leží napríklad
v metóde skladaní sekvencíí GAML \cite{gaml}.

V práci Philippe, Salsona a kol. \cite{gk_arrays} sa spomína sedem typov otázok,
ktoré môžeme mať na nejakú sadu čítaní:

\begin{enumerate}
    \item Ktoré čítania obsahujú daný podreťazec $f$?
    \item Koľko čítaní obsahuje daný podreťazec $f$?
    \item Na ktorých pozíciach v čítaniach sa nachádza daný podreťazec $f$?
    \item Koľkokrát sa dohromady vyskytuje daný podreťazec $f$ v čítaniach?
    \item V ktorých čítaniach sa vyskytuje podreťazec $f$ iba raz?
    \item V koľkých čítaniach sa vyskytuje podreťazec $f$ iba raz?
    \item Na ktorých pozíciach v čítaniach sa nachádza podreťazec $f$, pričom nás
    zaujímajú iba tie čítania, ktoré ho obsahujú raz?
\end{enumerate}

Cieľom tejto práce bude navrhnúť
štruktúru, ktorá bude schopná efektívne odpovedať na niektoré z týchto otázok
a ktorá bude zároveň zaberať čo najmenej miesta v pamäti.

\section{Súvisiaca práca}

Prvou význačnou prácou pri indexácii čítaní sú tzv. Gk-Arrays \cite{gk_arrays}, ktoré
vyhľadávajú podreťazce s pomocou upravených sufixových polí nad zreťazením všetkých
čítaní.

Myšlienku Gk-arrays posunuli ďalej Välimäki a Rivals, keď prišli s komprimovanou
verziou \cite{comp_gk_arrays}.

S myšlienkovo odlišným riešením prišli Boža a Jursa. Namiesto zreťazenia čítaní v
ich CR-indexe \cite{cr_index} nájdu nadslovo všetkých čítaní a pomocou neho
odpovedajú na dotazy. V tejto práci rozvinieme podobnú myšlienku s tým, že budeme
hľadať mierne odlišné nadslovo.

\section{Definície}

Vrámci celej práce budeme považovať nulu za prirodzené číslo. Pod podslovom slova $s$
budeme rozumieť súvislú podpostupnosť jeho písmen. Jednou z centrálnych definícií bude
pojem $k$-nadslova.

\begin{defn}
    Nech $k > 1$ je celé číslo. Nech $\Sigma$ je nejaká abeceda a nech $S$ je množina
    slov dlhých aspoň $k$ nad abecedou $\Sigma$. Potom $k$-nadslovom množiny $S$ nazveme také slovo
    nad abecedou $\Sigma$, pre ktoré platí, že každé podslovo nejakého slova z $S$ je
    zároveň podslovom $k$-nadslova.
\end{defn}

V tejto práci budeme taktiež používať orientované grafy. Pod pojmom orientovaný graf $G$ budeme
rozumieť dvojicu $(V, E)$, kde $V$ je množina vrcholov a $E$ je množina usporiadaných
dvojíc $(v_1, v_2)$, kde $v_1$ aj $v_2$ sú vrcholmi grafu. Budeme hovoriť, že hrana $e$ vedie
z vrchola $v_1$ do vrchola $v_2$, pokiaľ $e = (v_1, v_2)$.

Pre graf $G$ budeme používať $V(G)$ na označenie jeho množiny vrcholov a $E(G)$ na označenie jeho
množiny hrán. Pod množinou výstupných hrán
vrchola $v$ budeme rozumieť množinu $out(v) = \{ (v, v_2) | v_2 \in V(G) \}$ a podobne,
pod množinou vstupných hrán budeme rozumieť množinu $in(v) = \{ (v_1, v) | v_1 \in V(G) \}$.
Pre vrchol $v$ určíme jeho výstupný stupeň ako $d_O(v) = |out(v)|$ a jeho vstupný stupeň ako $d_I(v) = |in(v)|$.

V prípade, že budeme hovoriť o viacerých grafoch a z kontextu nie je jasné, na ktorý z nich
sa niektorý z pojmov vzťahuje, tento graf zadefinujeme v hornom indexe, napríklad
$d_O^{G_2}(v)$ bude označovať výstupný stupeň vrchola $v$ v grafe $G_2$.

Pod pojmom \emph{sled grafu $G$} budeme rozumieť ľubovoľnú striedavú postupnosť vrcholov
a hrán $v_1 e_1 v_2 e_2 \ldots v_{n-1} e_{n-1} v_n$, ktorá začína aj končí nejakým vrcholom
grafu $G$ a v ktorej každá hrana $e_i$ vedie z vrchola $v_i$ do vrchola $v_{i+1}$.

%TODO: Definicia k-nadslova sa musi vysporiadat so slovami kratsimi ako k.
%TODO: Definicia k-nadslova musi zarucit, ze S obsahuje aspon jedno slovo dlzky k.
%TODO: Definicia stupnov (delt) musi zadefinovat aj, v ktorom grafe.
%TODO: rank, (select?)
