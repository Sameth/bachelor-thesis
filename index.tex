\chapter{Indexácia $k$-nadslova}

V predošlej kapitole sme si popísali, ako vieme nájsť $k$-nadslovo k nejakej
množine reťazcov. V tejto kapitole si popíšeme rôzne typy dotazov, ktoré by
sme chceli mať na tieto reťazce a ako ich vieme implementovať s pomocou
nájdeného $k$-nadslova. Pripomíname, že všetky dotazy budú zisťovať nejakú
informáciu o podreťazcoch, ktoré sú dlhé nanajvýš $k$.

\section{Prítomnosť podreťazca}

Úplne najzákladnejšou otázkou, na ktorú chceme vedieť odpovedať, je prítomnosť
nejakého podreťazca dĺžky najviac $k$ vo vstupných reťazcoch. Ako sa neskôr ukáže,
vedieť odpovedať na takýto dotaz nám neskôr pomôže vybudovať dodatočné podporné
štruktúry, pomocou ktorých budeme vedieť odpovedať aj na zložitejšie dotazy.

Ako prvé je dôležité všimnúť si, že prítomnosť nejakého reťazca v $k$-nadslove
nám nezaručuje, že sa vyskytoval aj v nejakom z pôvodných reťazcov. Vezmime si
napríklad $k = 4$ a dva reťazce, \texttt{ACGA} a \texttt{ACGT}. Ich $k$-nadslovom
môže byť napríklad reťazec \texttt{ACGACGT}, ktorý napríklad obsahuje aj
podreťazec \texttt{CGAC}, ktorý sa nevyskytuje ani v jednom z pôvodných reťazcov.

Pre tento účel mierne modifikujeme algoritmus na hľadanie $k$-nadslova tak, aby nám
vedel pre každý znak výsledného $k$-nadslova povedať, či sa ním končí nejaká $k$-tica,
ktorá sa vyskytuje v niektorom pôvodnom reťazci. To docielime jednoducho tak, že
pre každú hranu, ktorú budeme mať v našom grafe, si zapamätáme, či reprezentuje
nutnú hranu, inými slovami povedané, či sa $k$-tica, ktorú reprezentuje, vyskytuje
v niektorom zo vstupných reťazcov.

Pri prepisovaní sledu na konkrétny reťazec si budeme zároveň vytvárať pole
pravdivostných hodnôt, do ktorého pre každý znak, ktorý pripisujeme na koniec
generovaného $k$-nadslova, uložíme hodnotu $Pravda$, ak práve pridávaný znak
prichádza z nutnej hrany. Prvých $k-1$ znakov, preklad prvého vrchola, dostane
hodnotu $Pravda$, keďže každý vrchol predstavuje $k-1$-ticu znakov, ktorá sa
nachádzala v niektorom vstupnom slove. Toto pole budeme nazývať \emph{poľom výskytov}.

Nakoniec potrebujeme pre každý dotazovaný podreťazec, kde sa nachádza v $k$-nadslove.
Pre tento účel využijeme štruktúru FM-index \cite{fm_index}, konkrétne jeho implementáciu
z knižnice SDSL \cite{sdsl}. Pomocou FM-indexu vieme zistiť, kde sa začínajú všetky
výskyty nejakého podreťazca v $k$-nadslove v krátkom čase aj napriek tomu, že
zaberá iba málo pamäte.

Na dotazovaný reťazec $r$ teda nájdeme všetky jeho začiatky v $k$-nadslove. Ak
v poli výskytov nájdeme aspoň pre jeden koniec výskytu hodnotu $Pravda$, odpovieme
tiež $Pravda$, v opačnom prípade odpovieme $Nepravda$.

\section{Počet výskytov podreťazca}

Podobne, ako zisťovanie prítomnosti nejakého podreťazca, môžeme vyriešiť aj podporu
pre dotazy, koľkokrát sa daný podreťazec nachádza vo vstupných reťazcoch. Jediný rozdiel
bude v tom, že si pre každú nutnú hranu budeme pamätať, koľkokrát sa vyskytovala jej
zodpovedajúca $k$-tica písmen na vstupe, pre nepovinné hrany si budeme pamätať nulu.

Taktiež podobne ako pri dotazoch na prítomnosť si pre každý znak v $k$-nadslove
budeme pamätať jednu hodnotu -- koľko $k$-tic zo vstupu sa končí týmto znakom. Toto
pole budeme ďalej nazývať poľom počtu výskytov.

Spracovávanie dotazu potom vyzerá podobne, ako pri zisťovaní prítomnosti. Nájdeme
si všetky výskyty dotazovaného reťazca v FM-indexe. Spomedzi týchto výskytov nás
bude zaujímať ten, u ktorého poslednému znaku zodpovedá nenulové číslo
v poli počtu výskytov a toto číslo vrátime ako odpoveď. 

Keďže každú $k$-ticu pri konštruovaní
reprezentujeme ako jednu nutnú hranu, práve jeden výskyt zodpovedá tejto hrane a
je nenulový, ostatné musia byť nulové. Dôležité je však poznamenať, že takéto
riešenie dotazu nemusí dávať správnu odpoveď pre dotazované reťazce kratšie ako $k$.

Aby toto pole čísel nezaberalo oveľa viac pamäte, než samotný FM-index, uložíme
si ho skomprimované v štruktúre, ktorá je implementovaná v SDSL \cite{sdsl}.
Experimenty ukázali, že vhodne skomprimované pole zaberá zhruba toľko miesta,
čo FM-index. 
