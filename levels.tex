\chapter{Požiadavky na riešenie}

Problém, ktorý riešime v tejto práci sa dá rozdeliť na tri základné úrovne podľa toho,
ako veľmi berieme do úvahy požadovanú funkcionalitu a povahu reálnych dát. Pôjde o:
\begin{enumerate}
    \item Hľadanie $k$-nadslova
    \item Indexovanie do $k$-nadslova
    \item Tolerancia chýb v dátach
\end{enumerate}
V tejto kapitole sa pozrieme na každú z týchto úrovní. Popíšeme spôsob, akým zasahujú
do riešenia, ako ovplyvňujú ťažkosť a zložitosť a popíšeme si, ako problém danej
úrovne budeme riešiť.

\section{Hľadanie $k$-nadslova}

Prvým a najmenej zložitým problémom je hľadanie najkratšieho $k$-nadslova. Na pripomenuie,
ide o nadslovo, ktoré obsahuje všetky podslová dĺžky $k$ zo zadaných slov.

Mohlo by sa zdať, že ide o slabšiu verziu problému hľadania
spoločného nadslova, keďže ľubovoľné nadslovo nejakej množiny reťazcov bude zároveň
$k$-nadslovom tejto množiny reťazcov, ale ľahko vidno, že naopak to neplatí.
Ukazuje sa však, že rovnako, ako problém hľadania najkratšieho nadslova, je aj
problém hľadania najkratšieho $k$-nadslova $\mathcal{NP}$-ťažký.

\subsection{$\mathcal{NP}$-ťažkosť problému}

Rozhodovacia verzia problému najkratšieho spoločného nadslova je pomocou redukcie
z problému Hamiltonovskej cesty v orientovanom grafe s obmedzením na stupne vrcholov
a následnej redukcie na tento pomocný problém z problému Hamiltonovskej cesty v jednoduchom
grafe, $\mathcal{NP}$-úplný, pokiaľ je splnená aspoň jedna z nasledujúcich podmienok\cite{superstring}:
\begin{itemize}
    \item Veľkosť abecedy, nad ktorou máme zadané slovo je neobmedzená
    \item Veľkosť abecedy je aspoň $2$ a existuje $h \ge 1$ také, že každý
          vstupný reťazec má dĺžku $\lceil h \cdot log_2 (S) \rceil$, kde $S$
          je celková dĺžka vstupných slov.
\end{itemize}

Pokiaľ by sme vedeli efektívne riešiť rozhodovací problém najkratšieho $k$-nadslova,
na vstupe s rovnako dlhými vstupnými reťazcami a $k$ rovným dĺžke týchto reťazcov by
bolo riešenie zhodné s riešením rozhodovacieho problému najkratšieho spoločného nadslova.
Preto ak by sme mali polynomiálny algoritmus, ktorý vie riešiť rozhodovací problém
najkratšieho $k$-nadslova, mali by sme zároveň polynomiálny algoritmus riešiaci problém
Hamiltonovskej cesty. Trieda takýchto vstupov spĺňa druhú zo spomenutých podmienok.

Z toho vyplýva $\mathcal{NP}$-ťažkosť rozhodovacieho problému
najkratšieho $k$-nadslova a tým pádom aj $\mathcal{NP}$-ťažkosť problému hľadania
najkratšieho $k$-nadslova.

\subsection{Abstrakcia}

Po zdôvodnení a dokázaní ťažkosti riešeného problému sa pozrieme na spôsob, akým ho budeme
riešiť. Ako prvé si objasníme abstrakciu problému, na ktorej postavíme celé riešenie.

Podobne ako pri dokazovaní ťažkosti problému si pomôžeme grafovou teóriou. Celú
sadu vstupných slov si budeme reprezentovať ako mierne upravený ohodnotený de Bruijnov graf stupňa $k - 1$ nad
abecedou $\{A, C, G, T\}$. Každej hrane symbolicky priradíme reťazec, ktorého dĺžka bude
vždy zároveň hodnotou hrany.
Naše riešenie, čiže nejaké spoločné $k$-nadslovo, budeme reprezentovať ako sled v tomto grafe.

Vrcholmi v tomto grafe budú všetky $(k - 1)$-tice písmen ktoré sa vyskytujú v aspoň jednom slove.
Z vrchola $\left(x_1 x_2 x_3 \ldots x_{k-1}\right)$ bude do vrchola $\left(x_2 x_3 \ldots x_k\right)$
viesť hrana, ktorú označíme ako \emph{nutnú},
práve vtedy, ak slovo $x_1 x_2 x_3 \ldots x_k$ je podslovom niektorého zo
vstupných slov. Každej \emph{nutnej} hrane priradíme jednoprvkový reťazec pozostávajúci z $x_k$, čiže
posledného znaku vo vrchole, do ktorého smeruje.

Môžeme si všimnúť, že v takomto grafe budú všetky $k$-tice písmen, ktoré sú
ako podslovo niektorého vstupného slova, reprezentované ako \emph{nutné} hrany. Ďalej
potrebujeme zaviesť pojem \emph{nepovinnej} hrany.

Ak z vrchola $S_1$ nevedie \emph{nutná} hrana do vrchola $S_2$, tak z vrchola
$S_1$ vedie do vrchola $S_2$ \emph{nepovinná} hrana. Pre zadefinovanie hodnoty
hrany a priradeného reťazca potrebujeme ešte jeden pojem.

\begin{defn}
Nech $s_1$ a $s_2$ sú reťazce znakov a nech $z$ je najdlhší reťazec taký, že
$$\exists x, y \in \{A, C, G, T\}^*: s_1 = xz \wedge s_2 = zy$$
Potom prekryvom $s_1$ a $s_2$ označíme reťazec $z$ a dokončením $s_1$ v $s_2$ označíme
taký reťazec $y$, ktorý spĺňa $s_2 = zy$.
\end{defn}

\emph{Nepovinnej} hrane z vrchola $S_1$ do vrchola $S_2$ priradíme dokončenie $S_1$ v
$S_2$. Pripomíname, že hodnotou tejto hrany bude dĺžka priradeného reťazca.
Takto skonštruovaný graf k množine vstupných slov $S$ budeme označovať ako
\textbf{\boldmath$k$-graf množiny \boldmath$S$}.

Slovo prislúchajúce sledu v $k$-grafe množiny $S$ získame ako zreťazenie $zr$, kde $z$ je
označenie prvého vrchola sledu a $r$ sú pospájané reťazce priradené hranám v takom
poradí, v akom sa tieto hrany nachádzajú v slede.

Formálnejšie zapísané, nech $slovo(V)$ označuje reťazec dĺžky $k-1$, označenie vrchola
$V$ a $slovo(e)$ označuje reťazec priradený hrane $e$. Potom $slovo(w)$, slovo
prislúchajúce sledu $w = V_1 e_1 V_2 e_2 \ldots e_{n-1} V_n$ skonštruujeme
ako 
$$slovo(w) = slovo(V) \cdot slovo(e1) \cdot slovo(e2) \cdots slovo(e_{n-1}).$$

Sled v $k$-grafe množiny $S$ budeme volať \emph{korektný}, ak obsahuje všetky \emph{nutné} hrany.
Teraz si dokážeme, že slovo prislúchajúce \emph{korektnému} sledu v $k$-grafe množiny $S$ je
$k$-nadslovom množiny $S$. K tomu budeme potrebovať ešte jednu pomocnú lemu.

\begin{lema}
    Nech $V_1 e_1 V_2 e_2 \ldots e_{n-1} V_n$ je sled v $k$-grafe nejakej množiny $S$.
    Potom sa reťazec $r$ prislúchajúci tomuto sledu končí $(k-1)$-ticou znakov zhodnou
    s označením vrchola $V_n$.
\end{lema}

\begin{proof}
    Matematickou indukciou na $n$, dĺžku sledu.
   
    $1^0:$ Ak $n$ je rovné $1$, sled obsahuje iba jeden vrchol. Podľa konštrukcie slova
           priradeného sledu bude toto slovo totožné s označením prvého a zároveň aj
           posledného vrchola v slede.

    $2^0:$ Nech $w_1 = V_1 e_1 \ldots V_{j-1}$ a $w_2 = V_1 e_1 \ldots V_j$. Podľa konštrukcie slova $slovo(w_2)$ platí
           $$ slovo(w_2) = slovo(V_1) \cdot slovo(e_1) \cdots slovo(e_{j-1}) = slovo(w_1) \cdot slovo(e_{j-1})$$
           Ďalej potrebujeme rozobrať dva prípady:
           \begin{enumerate}
            \item Ak je hrana $e_{j-1}$ \emph{nutná}, tak platí
                  $$ \exists x_1, x_2 \ldots x_k \in \{A, C, G, T \}: V_{j-1} = (x_1, \ldots , x_{k-1}) \wedge V_j = (x_2, \ldots , x_k)$$
                  a podľa konštrukcie hrany a priradenia jej reťazca, $slovo(e_{j-1}) = x_n$.
                  Z indukčného predpokladu vyplýva, že $\exists q \in \{ A, C, G, T\}^*: slovo(w_1) = q \cdot x_1 \cdot x_2 \cdots x_{k-1}$. Keď to
                  spojíme dohromady, dostaneme
                  $$ slovo(w_2) = slovo(w_1) \cdot slovo(e_{j - 1}) = q \cdot x_1 \cdots x_{k - 1} \cdot slovo(e_{j - 1}) = $$
                  $$ = q \cdot x_1 \cdots x_{k - 1} \cdot x_k = q' \cdot x_2 \cdots x_k = q' \cdot slovo(V_j),$$
                  kde $q' = qx_1$.

            \item Ak je hrana $e_{j - 1}$ \emph{nepovinná}, tak z konštrukcie \emph{nepovinnej} hrany a jej priradeného reťazca
                  vyplýva:
                  $$ \exists x, y, z \in \{ A, C, G, T\}^* : slovo(V_{j - 1}) = xz \wedge slovo(V_{j}) = zy \wedge slovo(e_{j - 1}) = y.$$
                  Opäť z indukčného predpokladu vyplýva $ \exists q \in \{A, C, G, T\}^*: slovo(w_1) = q \cdot slovo(V_{j-1}) = q \cdot xz$.
                  Keď to poskladáme dohromady, dostaneme
                  $$ slovo(w_2) = slovo(w_1) \cdot slovo(e_{j - 1}) = q \cdot x \cdot z \cdot slovo(e_{j - 1}) = q \cdot x \cdot z \cdot y = $$
                  $$ = q \cdot x \cdot slovo(V_j).\qedhere$$

           \end{enumerate}
\end{proof}

\begin{veta}
    Nech $W$ je \emph{korektný} sled v $k$-grafe množiny $S$. Potom $slovo(W)$ je $k$-nadslovom množiny slov $S$.
\end{veta}

\begin{proof}
Vezmime si ľubovoľné slovo $w$ také, že $|w| = k$ a $w$ je podslovom nejakého slova z $S$. Nech $x_1, x_2, \cdots x_k$
sú znaky tohto slova v poradí. Potom sa v grafe vyskytujú vrcholy $V_1 = (x_1, x_2, \cdots, x_{k-1})$ a $V_2 = (x_2, x_3, \cdots x_k)$ také,
že z $V_1$ ide hrana $e$ do $V_2$ a $e$ je \emph{nutná}. Z toho vyplýva, že $W$ = $U_1 V_1 e V_2 U_2$, kde $U_1, U_2$ sú nejaké časti sledu.
Z predošlej lemy potom vyplýva, že $\exists q \in \{A,C,G,T \}^*: slovo(U_1 V_1) =  q \cdot x_1 \cdots x_{k - 1}$ a teda
$$slovo(W) = slovo(U_1 V_1 e V_2 U_2) = q \cdot x_1 \cdots x_{k - 1} \cdot slovo(e) \cdot slovo (V_2 U_2) = q \cdot w \cdot slovo(V_2 U_2).\qedhere$$
\end{proof}

\subsection{Vlastnosti abstrakcie}

Zatiaľčo každému korektnému sledu v abstrakcii prislúcha práve jedno $k$-nadslovo, nie
každé $k$-nadslovo vieme reprezentovať ako sled v našom grafe.

Ak si napríklad vezmeme ako vstupné slová \verb_ACG_ a \verb_CGA_ a $k$ rovné trom,
budeme mať v našom grafe tri vrcholy zodpovedajúce dvojiciam \verb_(AC)_, \verb_(CG)_ a
\verb_(GA)_. Hrany aj s označením, ktoré budeme používať ďalej: 
\begin{itemize}
    \item \emph{nutná} hrana $e_1$ z vrchola \verb_(AC)_ do \verb_(CG)_ s priradeným reťazcom \verb_G_,
    \item \emph{nutná} hrana $e_2$ z vrchola \verb_(CG)_ do \verb_(GA)_ s priradeným reťazcom \verb_A_,
    \item \emph{nepovinná} hrana $e_3$ z vrchola \verb_(AC)_ do \verb_(GA)_ s reťazcom \verb_GA_,
    \item \emph{nepovinná} hrana $e_4$ z vrchola \verb_(CG)_ do \verb_(AC)_ s reťazcom \verb_AC_,
    \item \emph{nepovinná} hrana $e_5$ z vrchola \verb_(GA)_ do \verb_(AC)_ s reťazcom \verb_C_,
    \item \emph{nepovinná} hrana $e_6$ z vrchola \verb_(GA)_ do \verb_(CG)_ s reťazcom \verb_CG_.
\end{itemize}

Celkom ľahko vidíme, že najkratšie možné $k$-nadslovo musí mať aspoň $4$ znaky. Vhodné $k$-nadslovo takejto
dĺžky naozaj existuje, napríklad \verb_ACGA_. V našom grafe ho vieme reprezentovať ako sled
$$\texttt{(AC)} e_1 \texttt{(CG)} e_2 \texttt{(GA)}$$

Ak sa ale pozrieme napríklad na slovo \verb_TTACGTTTCGATT_, neexistuje žiaden sled, ktorému prislúcha toto $k$-nadslovo,
keďže žiaden vrchol ani hrana neobsahujú znak \verb_T_. Ľahko ale vidíme, že toto slovo vieme skrátiť
vynechaním niektorých znakov na stále vyhovujúce $k$-nadslovo \verb_ACGCGA_. Ďalej si dokážeme, že toto 
pozorovanie vieme zovšeobecniť a ukážeme si, ako to vplýva na robustnosť našej abstrakcie problému.

\begin{veta}
Nech slovo $s$ je $k$-nadslovom množiny $S$. Potom ak neexistuje sled $W$ v $k$-grafe množiny $S$ taký, že
$slovo(W) = s$, existuje slovo $s'$ také, že $s'$ je $k$-nadslovom množiny $S$ a zároveň $|s'| < |s|$.
\end{veta}

